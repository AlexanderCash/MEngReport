%!TEX root = ../MEngProject.tex

%******************************************************************************************************
%*********************************** First Chapter ****************************************************
%******************************************************************************************************

\externaldocument{{../Chapter2/chapter2}}
\externaldocument{{../Chapter3/chapter3}}
\externaldocument{{../Chapter4/chapter4}}
\externaldocument{{../Chapter5/chapter5}}
\externaldocument{{../Chapter6/chapter6}}
\externaldocument{{../Chapter7/chapter7}}
\externaldocument{{../Appendix1/appendix1}}
\externaldocument{{../Appendix2/appendix2}}
\externaldocument{{../Appendix3/appendix3}}
\externaldocument{{../Appendix4/appendix4}}
\externaldocument{{../Appendix5/appendix5}}
\externaldocument{{../Appendix6/appendix6}}
\externaldocument{{../Appendix7/appendix7}}


\chapter{Introduction}  %Title of the First Chapter
\label{intro}

% **************************** Define Graphics Path **************************

\graphicspath{{Chapter1/Figs/}}

This report will document the planning, implementation, design, research, and testing that has all come together to form one large practical project. The area of interest is autonomous flight, and in particular the use of unmanned aerial vehicles. This project was a software based endeavour, and as such we shall discuss a number of the techniques employed to plan, create, and test code throughout. Where suitable, graphs and printouts will be used to illustrate and display findings, both positive and negative. 

This chapter is aimed at providing the reader with suitable background knowledge to provide context for this project, an understanding of the working environment, and an understanding for why the products of this work are desirable in the first place. Towards the end of this chapter the motivations for this project and the resulting objectives will be outlined in the simplest terms possible. 

Throughout the course of this work, there has been another project running concurrently with a number of overlapping themes and areas of interest. This other project is being completed by Mohammed Fahim Rahman and is looking into path optimisation. Where necessary this report shall mention the other project, in particular with regards to the sharing of one of the applications created as part of this work. 

%******************************************************************************************************
%******************************************************************************************************
\section{An Introduction to Unmanned Aerial Vehicles} 
\label{intro:UAVs}

The term Unmanned Aerial Vehicle (UAV) can be used to describe any form of aircraft that does not require a pilot on board. This term includes aircraft that are controlled remotely, autonomous vehicles capable of navigating themselves, and any aircraft that are a combination of the two. Other terms that may be familiar are ``drone'' and Unmanned Aerial System (UAS), which can be used interchangeably with UAV within the context of this project. 

UAVs come in a wide range of form factors but can be split into two main classes; fixed-wing UAVs and rotary wing UAVs. Fixed-wing UAVs are generally only capable of horizontal take-off and landing (HTOL), whilst rotary wing UAVs generally employ vertical take-off and landing (VTOL). There are, however, other forms of UAV which incorporate features of both main classes; for example, quadplane UAVS use rotary wing rotors to allow VTOL, but do the main bulk of their flying with forward facing rotors much like a fixed-wing UAV. Across these categories are a very wide range of UAV platforms, from hobbyist quadcopters to military attack vehicles, and many many others in between.

Throughout this project the UAVs in consideration are those that are suitable candidates for fully autonomous flight. In particular, this work is focussed on the types of UAV aimed at the hobbyist and civilian markets, as opposed to any form of military application. The usage scenario under consideration is that of aerial photography. With this in mind, this project will be considering fixed-wing UAVs, as they are the most suitable form factor for surveying large areas of land, due to their range and altitude capabilities. Henceforth when referring to UAVs, this paper is considering fixed-wing, small to medium sized aircraft. 
 

%******************************************************************************************************
%******************************************************************************************************
\section{UAVs for Aerial Photography} 
\label{intro:photography}

The first recorded use of aerial photography was in 1858 by a French balloonist and photographer known as ``Nadar'' at a height of 80 metres using a tethered hot air balloon \cite{Nadar}. Since then aerial photography has of course massively progressed, both in terms of the aerial vehicles in use and the cameras available. As such, it is now an activity that can be partaken in by organisations, companies, and individuals alike for a wide range of purposes. For example, aerial photography of course plays a very large role in intelligence gathering by government agencies and militaries, but can also be a hobby for otherwise earth-bound photography enthusiasts. 

Different applications will require different equipment and styles of photography; a hobbyist may want a wide panorama, whilst a mapping application will require a consistent vertical camera angle. The use of aerial photography for mapping or measurement is known as aerial photogrammetry, and will be the chosen field of interest for this project. Aerial photogrammetry is a specific subset of aerial photography in that it not only requires the capturing of images, but additional information about where the image was captured. For this project, we are looking at the use of aerial photogrammetry to create an image of the land below a UAV, either for mapping or general surveying purposes. In order to be able to do this, we need to be able to capture a set of images, and know the location of the UAV when the images were captured, to allow the images to be stitched together into one larger image of the ground beneath. In this scenario we can typically use GPS co-ordinates and a pre-planned imaging path to define which images relate to which areas of the ground below.


This project will primarily focus on the mapping of land beneath the UAV, and as such specific flight paths need to be defined before take-off. There are a number of typical flightpaths for surveying or mapping land, two forms are lawnmower pattern paths and spiral pattern paths. Fig. \ref{fig:simplelawnmower} shows the general form of a lawnmower pattern, and Fig. \ref{fig:simplespiral} shows the general form of a spiral pattern.

\begin{figure}[htbp!] 
\centering    
\includegraphics[width=0.7\textwidth]{SimpleLawnmower}
\caption[A simple lawnmower pattern flightpath]{The general form of a lawnmower pattern aerial imaging flightpath}
\label{fig:simplelawnmower}
\end{figure}

\begin{figure}[htbp!] 
\centering    
\includegraphics[width=0.7\textwidth]{SimpleSpiral}
\caption[A simple spiral pattern flightpath]{The general form of a spiral pattern aerial imaging flightpath}
\label{fig:simplespiral}
\end{figure}

Using these patterns, it is possible to define what shall be referred to henceforth as ``imaging paths'' or ``imaging runs''. These are sections of the flight path along which images of the ground below are being captured. Accordingly, whilst travelling these imaging paths, the position and orientation of the UAV are of the utmost importance. For the lawnmower pattern the imaging paths are the vertical sections of the path seen in Fig. \ref{fig:simplelawnmower}, whereas for the spiral path all sections of the flight plan are considered imaging paths. 

As the lawnmower pattern does not require the horizontal sections of the path in Fig. \ref{fig:simplelawnmower} to be used for capturing images, the precise route flown between the vertical paths is unimportant. Knowing this, the flightpath can be simply specified in terms of just the imaging paths, as can be seen in Fig. \ref{fig:imaginglawnmower}.

\begin{figure}[htbp!] 
\centering    
\includegraphics[width=0.7\textwidth]{ImagingLawnmower}
\caption[The lawnmower pattern displayed using only imaging paths]{A version of the lawnmower pattern, using only imaging paths}
\label{fig:imaginglawnmower}
\end{figure}

As the imaging paths shown in Fig. \ref{fig:imaginglawnmower} are separate of one another, it is possible to complete each imaging path in any order and in either direction. This does however require the user to plan accordingly and to record location data accurately so as to be able to stitch together the resulting photographs correctly. One of the resulting patterns that can be created from this is called the ``Zamboni pattern'', named after the route a Zamboni machine takes around an ice rink to smooth out the ice. A Zamboni pattern describes a path whereby the UAV does not traverse up and then down adjacent imaging paths, but instead performs a series of overlapping paths by travelling up one imaging path, and then down a path a few rows over. This is necessary for a Zamboni due to its large minimum turning radius, and would be equivalent to a UAV with a large turning radius being required to traverse a number of very close imaging paths. 

The required flightpath will have to be calculated based on a number of factors including:
\begin{itemize}
	\item The size of the area being photographed
	\item Changes in ground elevation over the chosen area
	\item The operational range of the UAV based on battery life or fuel supplies
	\item The desired resolution of the resulting images - higher detail images requires lower altitude photography
	\item The distance between imaging paths will be a function of the desired flying height and the angle of the lens on the camera
	\item The path to follow between the imaging paths will be a function of the turning performance of the UAV
	\item Environmental factors such as wind during the flight
\end{itemize}

These factors shall be considered and discussed later in this report where relevant, and have been used to inform decisions throughout the work of this project.

Throughout this work, the term ``mission'' will be used interchangeably with ``flightpath'', and describes a full, pre-planned flight path from take-off to landing. To define these missions we require particular software products that shall be discussed in Section \ref{intro:planner}.

%TODO mission planning process?
%******************************************************************************************************
%******************************************************************************************************
\section{ArduPilot} 
\label{intro:ardupilot}

ArduPilot, also known as ArduPilotMega (APM), is an open-source suite of autopilot products aimed at hobbyists and professionals alike. ArduPilot includes the ArduPlane, ArduCopter, and ArduRover autopilots, each aimed at aeroplane style UAVS, helicopter and multi-rotor UAVS, and ground based rover platforms respectively. As previously mentioned, this project is aimed at fixed-wing aeroplane UAVs, and as such this report is concerned with the ArduPlane product. 

ArduPilot has been around since late 2009 and was originally designed to operate using the Arduino development ecosystem. Since then, the code bases have evolved in both size and functionality, in part due to hardware improvements which will be discussed in Section \ref{intro:planner}. This has lead to both the software and hardware components of these ArduPilot autopilot solutions moving away from Arduino almost entirely.

ArduPilot grew out of an organisation called DIYDrones, which was set up as an online community for enthusiasts to discuss and create solutions for UAV use; this community still exists and operates as one of the many hubs for discussion regarding UAVs, autopilots, and route planning. In 2014, the group of people maintaining the ArduPilot project teamed up with The Linux Foundation to create a non-profit organisation called the Dronecode Project \cite{Dronecode}. The Dronecode Project is an effort to collaborate a number of open-source hardware and software products under one structure, which includes the ArduPilot project. In February 2016, it was announced that the core team behind ArduPilot would be creating their own non-profit organisation to best suit the needs of the users of the ArduPilot products, however at the time of writing this is still a work in progress \cite{ArduNonProfit}.

\subsection{ArduPlane}
\label{intro:arduplane}

ArduPlane is designed so that when paired with a suitable controller board, it enables autonomous flight on almost any fixed-wing UAV \cite{ArduPlaneHome}. To do this, the ArduPlane software is compiled to create the firmware, which is then programmed onto the controller board alongside the mission plan. Once the controller board is connected to all of the necessary control outputs and sensor inputs, the UAV must be configured and tuned for flight, and is then ready for take-off. The configuration and tuning are of course capabilities provided by ArduPlane, however are not relevant for discussion within the context of this project. ArduPlane handles both automatic take-off and landings, as well as providing the option to change between a number of flight modes. The flight modes available include fly-by-wire modes, an acrobatics mode, a training mode, and an automatic flight mode.

The ArduPlane code base uses a very large collection of libraries, of which the majority are also used by the ArduCopter and ArduRover products. This, combined with the fact that ArduPlane is designed to work with any fixed-wind UAV means that a lot of the code is very generic and reusable. The code base for the whole ArduPilot project currently consists of roughly 11,000 files in total (when including submodules), and the specific ArduPlane code (excluding any shared libraries or resources) is made up of only 37 files. The sheer number of shared files, generic coding style, and open-source nature of this project combine to mean that in many respects the code commenting and documentation is very lacking and unspecific. That being said, there are a number of instructional guides and references that can be found on the ArduPilot website for anyone looking to modify the code or get involved with contribution. In addition to this, there are a number of forums that can provide many answers to common problems. 

As well as there being three main autopilot products, there are also a large number of differing released versions and code branches of each of these products, which are often platform specific or hardware specific. The result of this is that there are a huge number of versions of the software, often with only slight differences between them.


\subsection{JSBSim}
\label{intro:jsbsim}

As the ArduPilot project is open source, there are a number of associated open-source utilities available to download for testing purposes. One of the main tools used in ArduPlane development is called JSBSim, an open-source Flight Dynamics Model (FDM) which can be used as part of a flight simulator \cite{jsbsim}. 

It is possible to create a Software In The Loop (SITL) simulator using JSBSim, enabling the simulation of UAV flights using ArduPlane without any requirement for additional hardware. This enables a wide range of test to be performed on any changes to ArduPlane without the need to have an actual UAV or a controller board. Fig. \ref{fig:jsbsim} shows the SITL simulator as a whole, where we have the command window, an output console, and a map view. The map view shows the planned mission and the actual path the UAV will follow in the given scenario, the console window displays output and communication messages from the UAV, and the command window allows control of the simulator.

\begin{figure}[htbp!] 
\centering    
\includegraphics[width=\textwidth]{JSBSim}
\caption[JSBSim simulator]{A view of the JSBSim based flight simulator, showing a command window, output console, and simulator map view}
\label{fig:jsbsim}
\end{figure}

This simulator forms the bulk of the testing for this project, and is used to map flight paths and record performance. Each simulated UAV flight generates a log in the form of a \textit{.BIN} file, which can then be analysed using the MAVExplorer utility.

\subsubsection{MAVExplorer}
\label{intro:mavexplorer}
MAVExplorer is a log analysis tool, provided in the form of a python script suitably named \textit{MAVExplorer.py}. This allows us to generate a map of the area over which the UAV has flown with white lines displaying the planned mission, and a green line plotting the actual route the UAV flew. An example of this can be seen in Fig. \ref{fig:mavexplorer}.

\begin{figure}[htbp!] 
\centering    
\includegraphics[width=\textwidth]{MAVExplorer}
\caption[MAVExplorer example]{The mapping view provided by MAVExplorer utilising log files from flight simulations}
\label{fig:mavexplorer}
\end{figure}

\subsection{MissionPlanner and APM Planner}
\label{intro:planner}

As previously mentioned, for autonomous UAV flight a mission needs to be planned before flight. There are a number of software products capable of this, however the main two used by the ArduPilot project are called APM Planner 2 and MissionPlanner. Both of these products are open source, however APM Planner 2 runs on Linux, Windows, and Mac OSX, whilst MissionPlanner is a Windows only application. Both products enable mission planning, however the MissionPlanner product is more feature rich, reliable, and quick to use. 

The mission plans generated using either of these programs are of the same format; a text file with a number of commands, parameters, and often GPS locations. This is very useful as it allows manual editing of these text files where necessary.

There are a wide range of commands that can be included in a mission including take off commands, land commands, and navigation commands. For this project the most relevant navigation commands are those which operate using waypoints. The first of these types of commands are loiter commands; the UAV can be instructed to loiter around a point at a specified radius for a number of rotations (complete circles around a point), a duration, or indefinitely. The other form of navigation command is simply an instruction to travel to a point in space, with the option to define how close to that point the UAV must travel. 

Fig. \ref{fig:missionplanner} shows a typical mission plan in MissionPlanner, where a simple lawnmower pattern is being plotted. 

\begin{figure}[htbp!] 
\centering    
\includegraphics[width=\textwidth]{MissionPlanner}
\caption[MissionPlanner desktop software]{The mission planning view of the Windows based MissionPlanner software}
\label{fig:missionplanner}
\end{figure}

Another feature of both APM Planner 2 and MissionPlanner is their ability to act as a Ground Control Station (GCS). In this role, these products act as tracking and monitoring systems for UAVs whilst they are in flight, enabling the logging and display of flight data in real time. 

%******************************************************************************************************
%******************************************************************************************************
\section{Autopilot Hardware} 
\label{intro:hardware}

As also mentioned, one of the key components for autonomous UAV flight using ArduPlane is the controller board which must be mounted on the UAV itself. There are a number of currently available boards that ArduPilot can run on, and over the years there have been many versions of these. These control boards are responsible for processing the mission plan and controlling the aircraft accordingly. During flight, the control board uses GPS readings alongside other sensor readings where available to direct the aircraft. It does this using the servos connected to the control surfaces, and by controlling the thrust generated from the propeller(s) or engine(s). The controller board can be connected other devices too, such as camera gimbals and parachute release devices, however these are not in use throughout this project. 

There are currently 4 supported controller boards that have an open hardware design, and 4 controller boards with closed hardware designs. Additional to this are a number of controller boards which are no longer supported, as they do not have the storage capabilities to load on the latest versions of the code. The most popular currently supported board is the Pixhawk board, which can be seen in Fig. \ref{fig:pixhawk}, and has an open hardware design. The APM 2.x series of controller boards, seen in Fig. \ref{fig:apm2}, used to be very popular however are no longer supported for the latest versions of the products.

\begin{figure}[htbp!] 
\centering    
\includegraphics[width=0.6\textwidth]{Pixhawk}
\caption[Pixhawk controller board]{The Pixhawk controller board, for which the hardware specification and design is openly available, from \cite{Pixhawk}}
\label{fig:pixhawk}
\end{figure}

It is possible to still use the APM 2.x boards as they are still available for purchase online and will work with older versions of the ArduPlane code. This may be beneficial in certain situations, as the Pixhawk board is considerably more expensive than an APM board. The presence of a controller board is not essential for this project, as we are able to use the SITL simulator described in Section \ref{intro:jsbsim} to simulate UAV flight without one. If one is present, it is possible to use a Hardware In The Loop (HITL) simulator to test changes to the code base by running them on a controller board. In this instance, the controller board replaces JSBSim whilst simulating flight.

\begin{figure}[htbp!] 
\centering    
\includegraphics[width=0.5\textwidth]{APM2}
\caption[APM 2.6 controller board]{The APM 2.6 controller board, which is no longer supported by the ArduPilot project, from \cite{Apm2}}
\label{fig:apm2}
\end{figure}

%******************************************************************************************************
%******************************************************************************************************
\section{Project Motivations and Objectives} 
\label{intro:objectives}

The inspiration for this project comes from its supervisor, Dr Andrew Pomfret. He was part of an expedition to map a glacier in Svalbard, Norway using a UAV equipped with a camera in order to measure any movement of the glacier over time. Although using a professional grade UAV and associated route planning software, he found a number of its features lacking. The most relevant issue was that planning a battery efficient route that would also capture high quality images was laborious and considerably time consuming. Following lengthy discussion between the author and Dr Pomfret, two primary objectives were defined.

\subsection{Objective 1: Efficient Route Planning}
\label{intro:obj1}
The first objective was to implement an efficient method for path planning. The solution should enable a user to plan a route that allows us to deliver high quality aerial photographs, using as little battery power as is reasonable. The term ``high quality'' aerial photographs is used within this project to describe captured images that are fit for purpose. This requires the UAV to be following the planned flight mission, displaying as little roll as is suitable for flying to ensure the camera affixed to the aircraft is suitably aimed at the ground beneath. Capturing high quality aerial photographs requires a route to be planned that will enable the UAV to be flying with the correct direction and attitude over the entire area that we want to photograph. This requires that the UAV is aligned with any imaging path before commencing flight along it, so as to ensure the UAV does not need to re-orient itself too much whilst capturing images.

A further requirement for this task is that the planning method should be user-friendly and calculate a desired route from standard waypoint definitions provided in the mission planning software.

Building on these aims, it is necessary for this system to work for flight in any environment, meaning it must be able to plan a route that works for the UAV in windy conditions. As such we must incorporate knowledge of the flying environment when planning the path, be that in real-time when flying or beforehand when planning, this shall be explored later.

\subsection{Objective 2: Route Following}
\label{intro:obj2}

The second objective follows on from the first, and requires that the autopilot code be extended to enable the UAV to fly any newly defined paths. ArduPlane must be changed to understand any new commands or path descriptions introduced by the first task, whilst ensuring the autopilot product does not lose any functionality or regress in any other way.

Additionally, this objective should work with the first to provide flight in windy conditions. This may be entirely a feature of the planning stage or it may be a feature of the autopilot implementation, this shall be explored later.

These objectives are explained at the highest level possible in a simple manner, avoiding overly technical descriptions and information. These objectives have been expanded upon during the planning process, which can be seen in Chapter \ref{planning}.


%******************************************************************************************************
%******************************************************************************************************
\section{Introduction Summary}
\label{intro:summary}

This chapter has provided some background information and contextual information, to enable the reader to fully understand the desire for completing this project. By discussing ArduPlane and the associated controller boards it is hoped that the reader is able to begin to envision the working requirements and tools available to undertake this project. 