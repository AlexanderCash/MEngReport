%!TEX root = ../MEngProject.tex

%******************************** Seventh Chapter *****************************************************
%******************************************************************************************************

\externaldocument{{../Chapter1/chapter1}}
\externaldocument{{../Chapter2/chapter2}}
\externaldocument{{../Chapter3/chapter3}}
\externaldocument{{../Chapter4/chapter4}}
\externaldocument{{../Chapter5/chapter5}}
\externaldocument{{../Chapter6/chapter6}}
\externaldocument{{../Appendix1/appendix1}}
\externaldocument{{../Appendix2/appendix2}}
\externaldocument{{../Appendix3/appendix3}}
\externaldocument{{../Appendix4/appendix4}}
\externaldocument{{../Appendix5/appendix5}}
\externaldocument{{../Appendix6/appendix6}}
\externaldocument{{../Appendix7/appendix7}}

\chapter{Conclusions}
\label{conclusion}

% **************************** Define Graphics Path **************************
\graphicspath{{Chapter7/Figs/}}


%******************************************************************************************************
%******************************************************************************************************
\section{Conclusions}
\label{conclusion:conclusion}

There has been a great deal learned throughout the course of this project with regards to both UAV flight planning, open source software products, and the ArduPilot project itself. It is a great shame that the project suffered so many setbacks, as with a bit more time, a more suitable path planning solution could have been developed which may have changed the findings of this report considerably. The final results were both disheartening and frustrating, as the ideas behind this project show promise but have not been able to be fully realised. 

As mentioned time and time again, it was well known prior to making any changes to ArduPlane that the path planning solution was not completely fit for purpose. As the work being conducted involved so many unknown factors (primarily the behaviours of ArduPlane and the UAV being simulated), the attempt to ``try it and see'' may have proven successful given more time. If the path planning application were modified to better consider the true turning behaviours of the UAV however, it definitely has potential to meet all of the requirements laid out for aligning with imaging paths. A more finalised solution will not be able to make use of ``true'' Dubins paths, although a slightly modified version that lengthens turning arcs where necessary may prove very effective. By calculating a turn based on a Dubins path but with allowances and tolerances for the roll rate of the UAV, the changes made to ArduPlane could be leveraged to form a thorough and comprehensive solution to the problem discussed here. Perhaps this work will be conducted at some point in the future; its a nice thought at least.

The changes made to ArduPlane, whilst admittedly rudimentary, served their purpose and fulfilled all of the user stories planned out. In some regards the additions to ArduPlane were in fact perfect; they met the requirements in the simplest possible way, without compromising any of the pre-existing behaviours of the autopilot. If this work was to be continued with the aim of incorporating it into an official version of ArduPilot, not all that much would need to be changed in the ArduPlane code. By performing banked turns using the UAV's own roll limits, the solution implemented is in theory suitable for any UAV which may be using ArduPlane. Of course further testing on a range of platforms would be required to verify this, although it is hard to imagine why this would not be the case.

The problems faced throughout this project have been numerous, irritating, and often tricky to overcome. The open-source nature of ArduPlane and the fact that over recent months it has undergone very large changes to its codebase are two features that have combined to create whole problems of their own. As the developers behind the project are not working on this for financial gain, there is little incentive to produce thorough and detailed documentation. The result of this has been a lot of failed attempts at various tasks based on incomplete or unsuitable documentation. The number of problems that were also addressed via forum posts is also worrying; although the community involved in this work is clearly very knowledgeable, it doesn't inspire confidence when cobbling together solutions from a variety of posts on a number of topics.

There have however, been two mistakes that are clearly mistakes and should have been addressed sooner. The decision to try and work on an older version of the product, simply with the intention of reducing the cost of any purchased hardware, was clearly misguided. In retrospect this project should have started out with the latest version of the code from the get-go, not least because the latest versions are almost entirely C\texttt{++}, as opposed to the unholy combination of C, C\texttt{++}, and Arduino files which comprise the older version of the product. The second mistake was a result of stubbornness; progress should have been shelved and greater efforts made to build the code in an IDE much sooner. In fairness however, there was no documentation nor forum posts that suggested getting the code to build in Eclipse on Linux was even a possibility. 

When all is said and done, the behaviour of stock ArduPlane is not ideal, however it far outperforms the current version of this new path planning and following system. It would be nice to see further work completed on this project, as even an extra few weeks could create something great. Who knows, it could even lead to a real UAV flying using this code!
